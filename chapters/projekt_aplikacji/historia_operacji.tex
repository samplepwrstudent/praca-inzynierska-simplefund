\section{Historia operacji}

Wszystkie zarejestrowane operacje można przeglądać w uporządkowanej formie tabelarycznej. Widok ten został zrealizowany z wykorzystaniem biblioteki Material UI oraz wewnętrznego komponentu bazowego \textit{BaseTable}, który narzuca jednolitą stylistykę i zachowanie tabel w całej aplikacji. Jego wygląd przedstawiono na Rysunku \ref{fig:historia_operacji}. Widok umożliwia także usuwanie operacji oraz sortowanie tabeli względem wskazanej kolumny. 

\begin{figure}[h]
    \centering
    \includegraphics[width=0.8\linewidth]{images/historia_operacji.png}
    \caption{Tabela historii operacji}
    \label{fig:historia_operacji}
\end{figure}

Tabela historii operacji opiera się na mechanizmie kolumn typu \textit{GridColDef}, pochodzącym z komponentu tabeli danych MUI. Każda kolumna opisuje sposób prezentacji wybranego atrybutu operacji (np. daty, typu, kwoty, instrumentu finansowego), a definicje kolumn są następnie przekazywane do komponentu \textit{BaseTable}. Dzięki temu logika konfigurowania kolumn pozostaje lokalna dla danego widoku, natomiast powtarzalne elementy wyglądu oraz obsługi interakcji są współdzielone.


Na Listingu \ref{lst:columns-quantity} przedstawiono fragment kodu, który definiuje kolumny tabeli historii operacji. Zastosowana w nim logika warunkowa decyduje, czy dana kolumna powinna prezentować wartość w zależności od typu operacji.


W przypadku depozytów oraz lokat pola związane z instrumentami giełdowymi, takie jak \texttt{symbol} czy \texttt{wolumen}, zwracają wartość \textit{-}, co pozytywnie wpływa na czytelność i przejrzystość tabeli.

\begin{listing}[h]
\begin{minted}[fontsize=\small, linenos]{typescript}
const columns: GridColDef[] = [
    {
        field: 'quantity',
        headerName: 'Wolumen',
        align: 'left',
        headerAlign: 'left',
        type: 'number',
        flex: 0.8,
        minWidth: 100,
        valueGetter: (value, row: Operation) => 
            getStockOnlyValue(value, row.assetType),
        valueFormatter: formatNumberOrDash,
    },
];
\end{minted}
\caption{Konfiguracja kolumny \texttt{Wolumen} w tabeli danych}
\label{lst:columns-quantity}
\end{listing}