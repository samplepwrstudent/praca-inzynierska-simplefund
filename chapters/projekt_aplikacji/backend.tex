\section{Serwis Backend}

Warstwa backendowa aplikacji została zrealizowana jako lekki serwis REST API \cite{ovh2020restapi} zbudowany w technologii Flask. Aplikacja została wdrożona na platformie Heroku, która oferuje bezpłatny hosting dla aplikacji webowych dla studentów wraz z automatycznym zarządzaniem infrastrukturą oraz skalowalność. Platforma Heroku automatycznie wykrywa aplikacje Pythonowe na podstawie pliku \texttt{requirements.txt} i konfiguruje środowisko uruchomieniowe zgodnie z zadeklarowanymi zależnościami.

\begin{figure}[h]
    \centering
    \includegraphics[width=0.8\linewidth]{images/widok_panelu_heroku.png}
    \caption{Widok z panelu Heroku}
    \label{fig:placeholder}
\end{figure}

Zawartość pliku \texttt{Procfile} przedstawiona na Listingu \ref{lst:procfile} definiuje typy procesów oraz komendy uruchamiające aplikację. W przypadku omawianego projektu plik ten zawiera jedną linię określającą proces typu \texttt{web}, co wskazuje na aplikację webową obsługującą ruch HTTP.

\begin{listing}[h]
\begin{minted}[fontsize=\small]{text}
web: gunicorn app:app
\end{minted}
\caption{Zawartość pliku \texttt{Procfile}}
\label{lst:procfile}
\end{listing}

Powyższa konfiguracja instruuje Heroku, aby uruchomił serwer WSGI Gunicorn z modułem \texttt{app} oraz instancją aplikacji Flask o nazwie \texttt{app}. Gunicorn stanowi produkcyjny serwer HTTP dla aplikacji Pythonowych, zapewniający lepszą wydajność i stabilność w porównaniu do wbudowanego serwera deweloperskiego Flask.

Plik \texttt{requirements.txt} (Listing \ref{lst:requirements}) jest konieczny do poprawnego wdrożenia aplikacji w serwisie Heroku. Specyfikuje on wszystkie zależności projektu wraz z konkretnymi wersjami bibliotek, \cite{danesawszedzie2024requirements} co gwarantuje replikowalność środowiska uruchomieniowego.

\begin{listing}[h]
\begin{minted}[fontsize=\small]{text}
Flask==3.0.0
Flask-CORS==4.0.0
yfinance==0.2.66
gunicorn==21.2.0
\end{minted}
\caption{Zależności projektu zdefiniowane w pliku \texttt{requirements.txt}}
\label{lst:requirements}
\end{listing}


Główną funkcjonalnością serwisu backendowego jest pobieranie aktualnych danych giełdowych za pośrednictwem biblioteki \texttt{yfinance}. Biblioteka ta stanowi nieoficjalny pośrednik do API Yahoo Finance, umożliwiający programowy dostęp do historycznych oraz bieżących danych rynkowych dla różnych instrumentów finansowych. 

\begin{listing}[h]
\begin{minted}[fontsize=\small, linenos]{python}
@app.route('/stock', methods=['GET'])
def get_stock_info():
    ticker = request.args.get('ticker')
    # ...
    try:
        stock = yf.Ticker(ticker)
        hist = stock.history(period='1d')
        # ...
        last = hist.iloc[-1]
        
        stock_data = {
            'ticker': ticker.upper(),
            'currentPrice': round(float(last['Close']), 2),
            'open': round(float(last['Open']), 2),
            'dayHigh': round(float(last['High']), 2),
            'dayLow': round(float(last['Low']), 2),
            'volume': int(last['Volume']),
        }
        
        return jsonify(stock_data)
        # ...
\end{minted}
\caption{Fragment endpointu pobierającego dane giełdowe}
\label{lst:stock-endpoint}
\end{listing}

Funkcja przyjmuje parametr \texttt{ticker} z URL, pobiera historyczne dane za pomocą metody \texttt{history()}, a następnie ekstrahuje ostatni dostępny rekord zawierający najbardziej aktualne informacje. Dane są następnie formatowane i zwracane jako obiekt JSON zawierający cenę bieżącą, cenę otwarcia, wartości ekstremalne oraz wolumen obrotu. Jej sposób działania przedstawiono na Listingu \ref{lst:stock-endpoint}.


Cross-Origin Resource Sharing\cite{assis2023implementation} (CORS) stanowi mechanizm bezpieczeństwa przeglądarek internetowych, który kontroluje dostęp do zasobów serwera z różnych źródeł. Domyślnie przeglądarki blokują żądania HTTP wykonywane przez kod JavaScript do domen innych niż ta, z której pochodzi strona internetowa. Jest to część polityki Same-Origin Policy, mającej na celu ochronę przed atakami typu Cross-Site Request Forgery. 

W kontekście przedstawianej aplikacji frontend jest hostowany na platformie Firebase Hosting pod domeną \texttt{simplefund.me}, podczas gdy backend działa na Heroku pod inną domeną. Bez odpowiedniej konfiguracji CORS przeglądarka blokowałaby wszystkie żądania z aplikacji frontendowej do API backendowego. Skonfigurowanie pozwalające wybranym domeną na wysyłąnie zapytań do serwisu backendowego przedstawia Listing \ref{lst:cors-config}.

\begin{listing}[h]
\begin{minted}[fontsize=\small, linenos]{python}
CORS(app, resources={
    r"/*": {
        "origins": [
            'https://simple-fund-cf708.web.app',
            'https://simple-fund-cf708.firebaseapp.com',
            'https://simplefund.me'
        ]
    }
})
\end{minted}
\caption{Konfiguracja CORS z ograniczeniem do zaufanych domen}
\label{lst:cors-config}
\end{listing}
