\section{Diagram C4 aplikacji}
W celu przejrzystego przedstawienia architektury projektowanej aplikacji zdecydowano się na wykorzystanie diagramów obrazujących zarówno strukturę systemu, jak i organizację danych. Wizualizacja tych elementów stanowi istotny etap dokumentacji technicznej, ponieważ ułatwia zrozumienie zależności pomiędzy komponentami, sposobu przepływu informacji oraz logiki przechowywania danych.

Diagram C4 (Context, Containers, Components, Code) stanowi hierarchiczny sposób dokumentowania architektury systemów informatycznych, zaproponowany przez Simona Browna. \cite{grabis2025empirical} Zakłada on prezentację struktury systemu na czterech poziomach szczegółowości: kontekst systemowy (relacje z użytkownikami i systemami zewnętrznymi), kontenery (główne elementy techniczne aplikacji), komponenty (moduły wewnątrz kontenerów) oraz kod (szczegóły implementacyjne na poziomie klas i funkcji).

W przypadku omawianej pracy zdecydowano się na prezentację tylko drugiego poziomu: kontenerów. Diagram \ref{fig:diagram_c4} przedstawia główne elementy techniczne aplikacji: frontend, backend oraz zewnętrzne usługi, wraz z relacjami między nimi. 





%\section*{Diagram bazy danych Firestore}
%Aplikacja wykorzystuje Firestore - bazę danych typu NoSQL, oferowaną przez platformę Firebase. W przeciwieństwie do relacyjnych baz danych, Firestore organizuje dane w postaci kolekcji dokumentów, które mogą zawierać zagnieżdżone struktury oraz pola o różnych typach.

%Diagram \ref{fig:diagram_bazy} przedstawia strukturę bazy danych aplikacji, obejmującą kolekcje związane z użytkownikami, operacjami finansowymi, cenami akcji oraz symbolami giełdowymi. Każda kolekcja została opisana wraz z kluczowymi polami, typami danych oraz logiką identyfikatorów dokumentów. Diagram odzwierciedla rzeczywisty model danych zaimplementowany w aplikacji, uwzględniając zarówno dane użytkowników, jak i globalne zasoby współdzielone między użytkownikami.


%\begin{figure}[H]
%    \centering
%    \includegraphics[width=1\linewidth]{images/diagram_bazy_danych.png}
%    \caption{Diagram bazy danych}
%    \label{fig:diagram_bazy}
%\end{figure}