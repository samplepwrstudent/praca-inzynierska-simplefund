\section{Mechanizmy logowania, rejestracji i autoryzacji użytkowników}

Zapewnienie bezpiecznego i ergonomicznego mechanizmu logowania oraz rejestracji użytkowników stanowi podstawowy element każdej współczesnej aplikacji internetowej. W tym przypadku mechanizm autoryzacji warunkuje dostęp do zasobów i funkcji aplikacji (historia wykonanych operacji), chroniąc dane użytkowników przed nieautoryzowanym dostępem.

\begin{figure}[h]
    \centering

    \begin{subfigure}{0.45\linewidth}
        \centering
        \includegraphics[height=7cm]{images/jasny_ekran_logowania.png}
    \end{subfigure}
    \hfill
    \begin{subfigure}{0.45\linewidth}
        \centering
        \includegraphics[height=7cm]{images/rejestracja.png}
    \end{subfigure}

    \caption{Widoki logowania i rejestracji}
    \label{fig:dwa_obrazki}
\end{figure}


Fundamentem systemu autoryzacji jest moduł Firebase Authentication, \cite{programistajava2025firebaseauth} zintegrowany z aplikacją za pomocą biblioteki klienckiej Firebase SDK. Proces konfiguracji polega na pobraniu kluczy konfiguracyjnych z projektu Firebase oraz przekazaniu ich do aplikacji, aby Firebase mógł poprawnie zweryfikować źródło żądania i umożliwić autoryzowany dostęp. Na Listingu \ref{lst:firebase-init} przedstawiono inicjalizację usług Firebase. Szczególną uwagę w kontekście autoryzacji należy zwrócić na obiekt \texttt{auth}, odpowiedzialny za wykonywanie operacji uwierzytelniania.

Dodatkowo, aplikacja umożliwia logowanie za pośrednictwem konta Google (Rysunek \ref{fig:dwa_obrazki}), wykorzystując mechanizm uwierzytelniania federacyjnego. \cite{wolniewicz2012federacyjne} Jest to metoda, w której proces identyfikacji użytkownika powierzany jest zewnętrznemu, zaufanemu dostawcy tożsamości (Google), dzięki czemu aplikacja nie musi samodzielnie przechowywać ani weryfikować danych logowania. Rozwiązanie to zwiększa bezpieczeństwo oraz ułatwia użytkownikom dostęp do systemu poprzez wykorzystanie istniejących kont.

\begin{listing}[h]
\begin{minted}[fontsize=\small, linenos]{typescript}
import { initializeApp } from 'firebase/app';
import { getAuth, GoogleAuthProvider } from 'firebase/auth';
import { getFirestore } from 'firebase/firestore';

const firebaseConfig = {
    // Klucze firebase unikalne dla każdego projektu
};

const app = initializeApp(firebaseConfig);

export const auth = getAuth(app);
export const db = getFirestore(app);
export const googleProvider = new GoogleAuthProvider();
\end{minted}
\caption{Inicjalizacja usług Firebase}
\label{lst:firebase-init}
\end{listing}


Zalecaną praktyką jest weryfikowanie i odrzucanie nieprawidłowych bądź niekompletnych danych już na etapie ich wprowadzania przez użytkownika. W tym celu stworzono mechanizm walidacji oparty na bibliotece Zod,\cite{zod2025docs} definiującej schematy typowanych struktur danych, oraz React Hook Form, \cite{frontcave2025rhf} zarządzającym stanem formularzy i procesem ich obsługi. Zastosowanie tego podejścia pozwala na deklaratywną specyfikację reguł walidacji oraz automatyczne wyświetlanie komunikatów o błędach bezpośrednio w interfejsie użytkownika.


Schemat walidacji danych logowania wymaga poprawnego adresu poczty elektronicznej oraz hasła o minimalnej długości sześciu znaków, co przedstawiono na Listingu \ref{lst:login-validation}.

\begin{listing}[h]
\begin{minted}[fontsize=\small, linenos]{typescript}
const loginSchema = z.object({
  email: z.string()
    .min(1, 'Email jest wymagany')
    .email('Nieprawidłowy format email'),
  password: z.string()
    .min(6, 'Hasło musi mieć co najmniej 6 znaków'),
});
\end{minted}
\caption{Schemat walidacji dla formularza logowania}
\label{lst:login-validation}
\end{listing}

