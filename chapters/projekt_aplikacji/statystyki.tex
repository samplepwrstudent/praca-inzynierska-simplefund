\section{Statystyki portfela i wizualizacja danych}

Sekcja statystyk oraz wizualizacji danych ma na celu dostarczenie inwestorowi kompleksowych informacji o kondycji jego portfela, umożliwiając świadome podejmowanie decyzji inwestycyjnych. System statystyk został zaprojektowany w sposób pozwalający na szybką ocenę opłacalności inwestycji oraz określenie podatności portfela na spadki wartości.

\begin{figure}[h]
    \centering
    \includegraphics[width=1\linewidth]{images/statystyki_liczbowe.png}
    \caption{Widok wskaźników finansowych}
    \label{fig:placeholder}
\end{figure}

Sekcja statystyk została zaprojektowana jako zestaw pięciu kart informacyjnych, prezentujących najważniejsze wskaźniki finansowe charakteryzujące portfel inwestycyjny. Każda karta wyświetla pojedynczy wskaźnik wraz z jego aktualną wartością, a wybrane wskaźniki są dodatkowo opatrzone ikonami informacyjnymi zawierającymi ich pełne nazwy i definicje.

\subsubsection{Wkład własny}

Wskaźnik wkładu własnego reprezentuje sumę wszystkich środków zainwestowanych przez użytkownika w portfel. Wartość ta jest obliczana na podstawie historii operacji kupna i sprzedaży aktywów, uwzględniając rzeczywiste przepływy kapitału. Dla każdego aktywa w portfelu obliczana jest wartość całkowita wydatków zgodnie ze wzorem:

\begin{equation}
W = \sum_{i=1}^{n} W_i
\end{equation}

gdzie $W$ oznacza wkład własny całkowity, a $W_i$ to wkład w pojedyncze aktywo.

\subsubsection{Wartość inwestycji}

Wartość inwestycji określa bieżącą wartość rynkową wszystkich aktywów znajdujących się w portfelu. Wskaźnik ten jest obliczany na podstawie aktualnych cen rynkowych instrumentów finansowych oraz pozycji inwestora w każdym z nich:

\begin{equation}
V = \sum_{i=1}^{n} q_i \cdot p_i^{current}
\end{equation}

gdzie $V$ oznacza wartość inwestycji, $q_i$ to liczba posiadanych jednostek aktywa $i$, a $p_i^{current}$ to aktualna cena rynkowa tego aktywa.

\subsubsection{Bilans}

Bilans portfela stanowi różnicę między wartością inwestycji a wkładem własnym, reprezentując zysk lub stratę wygenerowaną przez portfel:

\begin{equation}
B = V - W
\end{equation}

gdzie $B$ to bilans, $V$ to wartość inwestycji, a $W$ to wkład własny. Wartość dodatnia wskazuje na zysk, natomiast ujemna na stratę.

\subsubsection{Stopa zwrotu (ROI)}

Wskaźnik ROI\cite{ks2024roi} (ang. Return on Investment) określa procentową stopę zwrotu z inwestycji względem wkładu własnego:

\begin{equation}
ROI = \frac{B}{W} \cdot 100\%
\end{equation}

gdzie $ROI$ to stopa zwrotu, $B$ to bilans portfela, a $W$ to wkład własny. Wskaźnik ten pozwala na ocenę efektywności inwestycji w ujęciu procentowym, niezależnie od wartości bezwzględnych.

\subsubsection{Średnioroczna stopa zwrotu (CAGR)}

Wskaźnik CAGR\cite{mfiles2023cagr} (ang. Compound Annual Growth Rate)  reprezentuje średnią roczną stopę zwrotu przy założeniu reinwestowania zysków. Jest to szczególnie istotny wskaźnik dla długoterminowych inwestycji, ponieważ uwzględnia efekt składania zysków w czasie:

\begin{equation}
CAGR = \left(\left(\frac{V}{W}\right)^{\frac{1}{t}} - 1\right) \cdot 100\%
\end{equation}

gdzie $CAGR$ to średnioroczna stopa zwrotu, $V$ to wartość inwestycji, $W$ to wkład własny, a $t$ to czas trwania inwestycji, wyrażony w latach.

Obliczanie wskaźnika CAGR w systemie uwzględnia precyzyjne określenie czasu trwania inwestycji na podstawie daty pierwszej operacji oraz bieżącej daty. Listing~\ref{lst:cagr-calculation} przedstawia kod funkcji realizującej to obliczenie.

\begin{listing}[h]
\begin{minted}[fontsize=\small, linenos]{typescript}
    const currentValue = calculateCurrentValue(assets);
    const cagr = (Math.pow(currentValue / investment, 1 / years) - 1) * 100;

    return cagr;
\end{minted}
\caption{Fragment funkcji obliczającej wskaźnik CAGR}
\label{lst:cagr-calculation}
\end{listing}

\begin{figure}[h]
    \centering
    \includegraphics[width=0.8\linewidth, height=9cm, keepaspectratio]{images/wykres_kolowy.png}
    \caption{Wykres kołowy reprezentujący kompozycję portfela}
    \label{fig:wykres_kolowy}
\end{figure}

Wizualizacja kompozycji portfela została zrealizowana przy użyciu wykresu kołowego (Rysunek \ref{fig:wykres_kolowy}), składającego się z dwóch obwódek. Architektura ta pozwala na jednoczesne przedstawienie dwóch poziomów szczegółowości informacji o strukturze portfela.


Wewnętrzna obwódka wykresu przedstawia podział portfela na poszczególne aktywa, takie jak konkretne akcje, obligacje lub lokaty. Każdy segment odpowiada jednemu aktywowi, a jego wielkość jest proporcjonalna do wartości tego aktywa w całym portfelu. Zewnętrzna obwódka agreguje aktywa według ich typu, prezentując udział procentowy akcji, lokat oszczędnościowych oraz innych kategorii instrumentów finansowych w całkowitej wartości portfela.

Takie podejście umożliwia inwestorowi szybką ocenę zarówno dywersyfikacji portfela na poziomie klas aktywów, jak i identyfikację pojedynczych pozycji, które mogą stanowić znaczącą część wartości inwestycji. Dwupoziomowa struktura jest szczególnie przydatna przy ocenie ryzyka koncentracji, ponieważ pozwala zauważyć zarówno nadmierną ekspozycję na pojedyncze aktywo, jak i niewłaściwą alokację między klasami aktywów \cite{gieraltowska2011dywersyfikacja}.


System kolorowania segmentów wykresu opiera się na zasadzie gradacji jasności dla aktywów należących do tej samej kategorii. Każda kategoria aktywów ma przypisany bazowy kolor, a poszczególne aktywa w ramach tej kategorii otrzymują odcienie o różnej jasności, uporządkowane według wartości ich udziału w portfelu.

Funkcja \texttt{createShade}, przedstawiona na Listingu \ref{lst:create-shade}, generuje odcienie koloru bazowego poprzez modyfikację komponentu jasności w przestrzeni barw HSL (Hue, Saturation, Lightness).

\begin{listing}[h]
\begin{minted}[fontsize=\small, linenos]{typescript}
export const createShade = (
    hsl: string, 
    index: number, 
    total: number
): string => {
    // parsowanie formatu: "hsl(40, 95%, 48%)"
    const match = hsl.match(/hsl\((\d+),\s*(\d+)%,\s*(\d+)%\)/);
    if (!match) return hsl;

    const h = parseInt(match[1]);
    const s = parseInt(match[2]);
    const l = parseInt(match[3]);

    const maxLightness = 95;
    const softerFactor = 0.7;
    const range = maxLightness - l;
    const divider = total - 1 === 0 ? 1 : total - 1;

    const step = (range / divider) * softerFactor;

    const newL = Math.min(maxLightness, l + index * step);

    return `hsl(${h}, ${s}%, ${newL}%)`;
};
\end{minted}
\caption{Funkcja generująca odcienie koloru dla segmentów wykresu}
\label{lst:create-shade}
\end{listing}

Algorytm wyodrębnia składowe HSL z koloru bazowego, następnie oblicza zakres możliwych wartości jasności od wartości bazowej do maksymalnej wartości $95\%$. Parametr \texttt{softerFactor} o wartości $0.7$ zmniejsza intensywność gradacji, zapobiegając powstawaniu zbyt jasnych odcieni, które mogłyby być trudne do rozróżnienia na jasnym tle interfejsu. Wartość jasności dla kolejnych aktywów jest obliczana zgodnie ze wzorem:

\begin{equation}
L_{new} = \min\left(L_{max}, L_{base} + i \cdot \frac{(L_{max} - L_{base}) \cdot f}{n - 1}\right)
\end{equation}

gdzie $L_{new}$ to nowa wartość jasności, $L_{max}$ to maksymalna wartość jasności, $L_{base}$ to bazowa wartość jasności z koloru kategorii, $i$ to indeks aktywa po posortowaniu według wartości, $f$ to współczynnik łagodzący gradację, a $n$ to liczba aktywów w kategorii.

Aktywa w ramach każdej kategorii są sortowane malejąco według wartości, a następnie kolorowane od najciemniejszego odcienia dla największego aktywa do najjaśniejszego dla najmniejszego. Takie uporządkowanie kolorów dodatkowo wzmacnia wizualne rozróżnienie między głównymi pozycjami a mniejszymi składnikami portfela.
