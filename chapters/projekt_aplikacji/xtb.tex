\section{Import operacji od brokera inwestycyjnego XTB}

W ramach aplikacji opracowano funkcję importu historii transakcji z domu maklerskiego XTB. XTB (X-Trade Brokers) jest jednym z najpopularniejszych brokerów w Polsce, \cite{brokerchooser2025ranking} szeroko wykorzystywanym przez inwestorów detalicznych, co czyni go naturalnym wyborem jako pierwsze źródło danych transakcyjnych integrowanych z aplikacją. Integracja z XTB pozwala użytkownikowi w prosty sposób załadować istniejącą historię operacji bez konieczności ręcznego wprowadzania pojedynczych transakcji.

Interfejs użytkownika udostępnia dedykowany przycisk otwierający modal odpowiedzialny za import pliku z raportem XTB. Po jego aktywacji wyświetlane jest okno dialogowe, w którym użytkownik wskazuje plik w formacie \texttt{.xlsx} wyeksportowany z panelu klienta XTB. Modal (rysunek \ref{fig:modal_xtb}) przedstawia kluczowe informacje dotyczące wymaganego formatu pliku oraz etapów procesu przetwarzania danych. Zawiera również instrukcje dotyczące pobierania historii transakcji z platformy XTB oraz komunikaty diagnostyczne pojawiające się w sytuacjach wyjątkowych, takich jak brak możliwości odnalezienia w serwisie Yahoo Finance symbolu instrumentu widniejącego w raporcie XTB.

\begin{figure}[h]
    \centering
    \includegraphics[width=0.8\linewidth]{images/xtb_modal.png}
    \caption{Modal importu operacji z XTB}
    \label{fig:modal_xtb}
\end{figure}

Po wskazaniu pliku, aplikacja rozpoczyna proces jego przetwarzania. Pierwszym etapem jest odczyt zawartości pliku \texttt{.xlsx} w przeglądarce z wykorzystaniem biblioteki \texttt{FileReader}. Plik jest wczytywany jako ciąg binarny, co następnie pozwala bibliotece \texttt{XLSX} na jego interpretację jako skoroszytu programu Excel. Na potrzeby niniejszej aplikacji zdefiniowano funkcję \texttt{parseXTBFile}, widoczną na Listingu \ref{lst:parse-xtb-file}. której zadaniem jest wyodrębnienie odpowiedniego arkusza z raportu XTB oraz konwersja jego struktury tabelarycznej do formatu \texttt{JSON}, dogodnego do dalszego, krokowego przetwarzania.

\begin{listing}[h]
\begin{minted}[fontsize=\small, linenos]{typescript}
        const reader = new FileReader();

        reader.onload = (e) => {
            try {
                const data = e.target?.result;
                if (!data) {
                    throw new Error('Brak danych w pliku');
                }
                const workbook = XLSX.read(data, { type: 'binary' });
                // ...
                // Convert sheet to JSON
                const jsonData = XLSX.utils.sheet_to_json<XTBRawRow>(worksheet, {
                    defval: undefined, // Default value for empty cells
                    raw: false, // Return values as strings (easier date parsing)
                });
\end{minted}
\caption{Konwersja pliku \texttt{.xlsx} z raportem XTB na strukturę JSON}
\label{lst:parse-xtb-file}
\end{listing}

Kolejnym etapem jest przetwarzanie uzyskanej tablicy JSON wiersz po wierszu. Każdy wiersz jest interpretowany jako potencjalna operacja giełdowa, którą należy zweryfikować, oczyścić i przekształcić do wewnętrznego formatu aplikacji. Wygląd tego procesu w aplikacji przedstawia Rysunek \ref{fig:przetwarzanie_xtb}.

\begin{figure}[h]
    \centering
    \includegraphics[width=0.8\linewidth]{images/przetwarzanie_xtb.png}
    \caption{Wygląd modala w trakcie przetwarzania wierszy pliku}
    \label{fig:przetwarzanie_xtb}
\end{figure}

W procesie przekształcania każdy wiersz jest poddawany szeregowi filtrów i walidacji. W pierwszej kolejności eliminowane są wiersze nagłówkowe oraz metadane, które nie reprezentują rzeczywistych transakcji (np. wiersze zawierające nazwy sekcji raportu, dane klienta czy podsumowania salda). W kolejnym kroku wartości tekstowe są konwertowane na liczby, usuwane są symbole walut oraz inne znaki nienumeryczne, a daty są przekształcane do obiektów \texttt{Timestamp} zgodnych z używaną w projekcie bazą danych. Dodatkowo symbol instrumentu jest wyszukiwany w Serwisie Yahoo Finance, co pozwala na dalszą integrację z modułem pobierania notowań giełdowych.

\begin{figure}[h]
\centering
\texttt{Plik XLSX}
$\longrightarrow$
\texttt{Plik JSON}
$\longrightarrow$
\shortstack{\textit{walidacja}\\ \textit{filtrowanie}\\ \textit{wyłuskiwanie}}
$\longrightarrow$
\shortstack{\texttt{wewnętrzny}\\  \texttt{format}\\ \texttt{aplikacji}}

\caption{Proces przekształcania danych wejściowych}
\label{fig:proces-przeksztalcania}
\end{figure}

Pełny przepływ danych od arkusza do rekordów w bazie danych przedstawia Rysunek \ref{fig:proces-przeksztalcania}. Tak zaprojektowany przepływ przetwarzania pliku XTB zapewnia użytkownikowi możliwość automatycznego zaimportowania pełnej historii transakcji przy zachowaniu spójności z wewnętrznym modelem danych aplikacji. Dzięki wykorzystaniu pośredniej warstwy w postaci JSON oraz iteracyjnemu przetwarzaniu wierszy, możliwe jest łatwe rozszerzanie obsługi o kolejne typy raportów lub dodatkowe pola, bez konieczności ingerencji w podstawowy mechanizm importu.