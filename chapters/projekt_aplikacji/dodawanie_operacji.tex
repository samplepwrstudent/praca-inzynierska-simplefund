\section{Dodawanie operacji}

W aplikacji operacja stanowi pojedynczy zapis dotyczący ruchu finansowego użytkownika, takiego jak wpłata środków na rachunek, założenie lokaty bankowej czy zawarcie transakcji giełdowej. Każda operacja zawiera informacje niezbędne do późniejszego obliczania bieżącej wartości portfela, łącznych wpłat i wypłat, a także zysków i strat związanych z inwestycjami.

\begin{figure}[H]
    \centering
    \includegraphics[height=4cm, keepaspectratio, width=0.5\linewidth]{images/wybor_rodzaju_aktywa.png}
    \caption{Przycisk dodawania nowej operacji wraz z wyborem aktywa}
    \label{fig:wybor_aktywa}
\end{figure}

Interakcja użytkownika z funkcją dodawania operacji rozpoczyna się od przycisku akcji umieszczonego w obrębie widoku portfela. Przycisk ten pozostaje stale dostępny w prawym dolnym rogu pulpitu. Umieszczenie go w tym obszarze pozwala na szybkie przejście od obserwacji do działania. Po rozwinięciu przycisku użytkownik otrzymuje dostęp do wyboru rodzaju operacji (Rysunek \ref{fig:wybor_aktywa}), co umożliwia uruchomienie odpowiedniego okna modalnego, odpowiadającego konkretnemu rodzajowi aktywów.

\begin{figure}[h] 
\centering 
\includegraphics[height=7cm, keepaspectratio, width=0.75\linewidth]{images/modal_lokata.png} 
\caption{Modal dodawania lokaty} 
\label{fig:dodaj_lokate} \
\end{figure}

Po wybraniu opcji dodania depozytu otwierany jest formularz w oknie modalnym (Rysunek  \ref{fig:dodaj_lokate}). Składa się on z pól, w których użytkownik wypełnia dane powiązane z aktywem. Formularz korzysta z komponentów React Hook Form oraz schematów walidacji Zod, co zapewnia kontrolę nad poprawnością danych wejściowych. W momencie zatwierdzenia formularza dane są przekazywane do logiki odpowiedzialnej za zapis w bazie danych Firestore, a następnie wykorzystywane do aktualizacji łącznej wartości środków wpłaconych przez użytkownika.

Najbardziej rozbudowany jest modal służący do dodawania operacji na akcjach. Formularz umożliwia użytkownikowi określenie symbolu instrumentu giełdowego, liczby nabywanych lub zbywanych jednostek, ceny jednostkowej, waluty oraz daty transakcji. Podczas wprowadzania symbolu aplikacja wyświetla sugestie na podstawie listy wcześniej zdefiniowanych instrumentów, co przedstawia Rysunek \ref{fig:sugestie_symboli}, przechowywanych w kolekcji Firestore. Umożliwia to szybkie i precyzyjne odnalezienie właściwego aktywa.

Po wybraniu symbolu aplikacja podejmuje próbę pobrania aktualnych lub historycznych danych rynkowych. W tym celu wysyła zapytanie do bazy danych. Jeżeli informacje o symbolu już się w niej znajdują, nie ma konieczności wysyłania zapytania do serwisu backendowego. Dzięki takiemu mechanizmowi redukujemy liczbę połączeń z zewnętrznym API.

\begin{figure}[h]
    \centering
    \includegraphics[height=7cm, keepaspectratio, width=1\linewidth]{images/list_symboli.png}
    \caption{Aplikacja wyświetla sugestie symboli}
    \label{fig:sugestie_symboli}
\end{figure}


W przeciwnym wypadku aplikacja, poprzez zapytanie do backendu, wyszukuje symbol w serwisie Yahoo Finance, wyświetlając odpowiedni komunikat, jak na Rysunku \ref{fig:wyszukiwanie_symbolu}. Następnie, w przypadku sukcesu, zapisuje informacje w Firestore, aby mogły być one ponownie wykorzystane.

\begin{figure}[h]
    \centering
    \includegraphics[width=0.8\linewidth]{images/wyszukiwanie_symbolu.png}
    \caption{Wyszukiwanie symbolu}
    \label{fig:wyszukiwanie_symbolu}
\end{figure}

Jeśli symbol zostanie poprawnie zidentyfikowany, aplikacja uzupełni informacje o jego walucie oraz aktualnej wartości, a sam symbol zostanie dodany do listy dostępnych instrumentów. W przypadku braku danych użytkownik otrzyma ostrzeżenie o niemożności weryfikacji symbolu (Rysunek \ref{fig:nie_odnaleziono_synmbolu}), jednak nadal będzie miał możliwość dodania symbolu do portfela jako symbolu niestandardowego. W takim przypadku dane o aktualnej cenie tego symbolu będą musiały być uzupełniane ręcznie. 


\begin{figure}[h]
    \centering
    \includegraphics[width=0.8\linewidth]{images/ostrzezenie_customowy_symbol.png}
    \caption{Komunikat o braku odnalezienia symbolu}
    \label{fig:nie_odnaleziono_synmbolu}
\end{figure}


Dzięki opisanej logice dodanie operacji na akcjach skutkuje nie tylko zarejestrowaniem transakcji w kolekcji operacji, lecz także aktualizacją powiązanych danych o wycenach instrumentów, co wprost przekłada się na możliwość późniejszego obliczania bieżącej wartości portfela oraz wyników inwestycyjnych.

