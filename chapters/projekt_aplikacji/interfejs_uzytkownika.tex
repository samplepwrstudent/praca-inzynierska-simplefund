\section{Interfejs użytkownika}

Aplikacja została zaprojektowana z wykorzystaniem biblioteki Material UI w wersji 7, która jest zgodna z założeniami Material Design 3 \cite{material32025expressive}. Interfejs charakteryzuje się nowoczesnym, minimalistycznym wyglądem oraz spójnym systemem projektowania, który obejmuje zarówno aspekty estetyczne, jak i funkcjonalne (Rysunek \ref{fig:fragment_ekranu_glownego}).

\begin{figure}[h]
    \centering
    \includegraphics[width=1\linewidth]{images/fragment_ekranu_glownego.png}
    \caption{Fragment ekranu głównego}
    \label{fig:fragment_ekranu_glownego}
\end{figure}

System typografii w aplikacji opiera się na sześciu rozmiarach czcionek. Rozmiary te zostały zdefiniowane w jednostkach \texttt{rem} \cite{sandu2023remcss} (Listing \ref{lst:font-sizes}), co zapewnia skalowalność interfejsu i lepszą dostępność. Najmniejszy rozmiar wykorzystywany jest do wyświetlania dodatkowych informacji i etykiet, podczas gdy największy jest zarezerwowany dla głównych nagłówków sekcji. Styl czcionki jest domyślny dla interfejsu MUI - Roboto.

\begin{listing}[h]
\begin{minted}[fontsize=\small, linenos]{typescript}
export const fontSizes = {
    tiny: '0.75rem',     // 12px
    small: '0.875rem',   // 14px
    medium: '1rem',      // 16px
    big: '1.125rem',     // 18px
    large: '1.25rem',    // 20px
    xlarge: '1.5rem',    // 24px
};
\end{minted}
\caption{Definicja rozmiarów czcionek}
\label{lst:font-sizes}
\end{listing}

Jednolity system odstępów stanowi podstawę spójności wizualnej interfejsu. Zdefiniowano osiem poziomów odstępów (Listing \ref{lst:spacing}), od \texttt{xxs} (4 px) do \texttt{jumbo} (64 px), które są konsekwentnie stosowane do marginesów, paddingów oraz przerw między elementami. Najmniejsze odstępy wykorzystywane są do precyzyjnego rozmieszczenia elementów wewnątrz komponentów, podczas gdy największe oddzielają główne sekcje interfejsu, tworząc wyraźne strefy funkcjonalne.

\begin{listing}[h]
\begin{minted}[fontsize=\small, linenos]{typescript}
export const spacing = {
    xxs: '0.25rem',     // 4px
    xs: '0.5rem',       // 8px
    s: '0.75rem',       // 12px
    m: '1rem',          // 16px
    l: '1.5rem',        // 24px
    xl: '2rem',         // 32px
    xxl: '3rem',        // 48px
    jumbo: '4rem',      // 64px
};
\end{minted}
\caption{Zadeklarowane wartości odstępów}
\label{lst:spacing}
\end{listing}

Aby zwiększyć komfort korzystania z aplikacji, wprowadzono motywy jasny i ciemny. Użytkownik ma możliwość swobodnego przełączania między trybami za pomocą dedykowanego przycisku w menu nawigacyjnym. Domyślnie aplikacja uruchamia się w trybie ciemnym, a wybór użytkownika jest zapamiętywany w pamięci lokalnej przeglądarki i przywracany w kolejnych sesjach. Porównanie przykładowego komponentu w różnych motywach znajduje się na Rysunku \ref{fig:porownanie_motywow}.
\begin{figure}[H]
    \centering

    \begin{subfigure}{0.45\linewidth}
        \centering
        \includegraphics[height=7cm]{images/jasny_ekran_logowania.png}
    \end{subfigure}
    \hfill
    \begin{subfigure}{0.45\linewidth}
        \centering
        \includegraphics[height=7cm]{images/ciemny_ekran_logowania.png}
    \end{subfigure}

    \caption{Porównanie ekranów logowania dla różnych motywów}
    \label{fig:porownanie_motywow}
\end{figure}

 Motyw jasny wykorzystuje neutralne, wysokiej jasności tło, które umożliwia klarowne rozdzielenie warstw interfejsu oraz zapewnia wysoki kontrast tekstu podstawowego i drugorzędnego. Motyw ciemny oparty na niskich wartościach jasności sprzyja komfortowi pracy przy słabym oświetleniu, jednocześnie zachowując wyraźną hierarchię informacji dzięki odpowiednio dobranym poziomom kontrastu. W obu wariantach zastosowano spójną paletę barw akcentowych w odcieniach bursztynowo-złotym i intensywnie pomarańczowym, które podkreślają elementy interakcyjne oraz wzmacniają wizualną identyfikację systemu. 

Do definiowania kolorów  w interfejsie aplikacji został wykorzystany model barw  (Rysunek \ref{fig:hsl_model}). Jest to przestrzeń barw, w której kolor opisywany jest za pomocą trzech parametrów: odcienia (Hue, 0–360°), nasycenia (Saturation, 0–100\%) oraz jasności (Lightness, 0–100\%).\cite{zabka2020podstawowe}

Zastosowanie tego modelu umożliwia bardziej intuicyjne operowanie kolorami, a także ułatwia stworzenie wariantów jasnych i ciemnych poprzez odpowiednią regulację komponentu jasności, co sprzyja zachowaniu spójności wizualnej systemu.

\begin{figure}[h]
    \centering
    \includegraphics[width=0.8\linewidth]{images/hsl.png}
    \caption{Przestrzeń barw HSL}
    \label{fig:hsl_model}
\end{figure}
