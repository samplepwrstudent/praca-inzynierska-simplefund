\Chapter{Zastosowane technologie i narzędzia}

Dobór technologii w projekcie podporządkowano łatwości utrzymania, szybkiemu rozwojowi oraz połączeniu z usługami chmurowymi. Poniżej przedstawiono najważniejsze wykorzystane narzędzia wraz z uzasadnieniem ich wyboru.

\section{Warstwa frontendowa}

Warstwę frontendową zrealizowano w oparciu o React 18, TypeScript, Material UI 7 oraz Firebase Hosting. React wybrano ze względu na szeroką dostępność materiałów edukacyjnych, modularność oraz bogaty ekosystem, w pełni kompatybilny z pozostałymi używanymi narzędziami\cite{edge1s2025angularreact}. Alternatywne rozwiązania, takie jak Angular czy Vue, uznano za mniej korzystne ze względu na większą złożoność lub mniejszą dostępność zasobów.

TypeScript zwiększył niezawodność i czytelność kodu dzięki statycznemu typowaniu, co jest istotne przy tworzeniu większych projektów\cite{mkedziora2024typescript}. Choć wymaga przestrzegania bardziej rygorystycznych zasad niż JavaScript, w zamian zapewnia lepszą jakość oraz przewidywalność kodu.

Material UI dostarczył gotowy, spójny zestaw komponentów zgodnych z Material Design 3, co znacząco przyspieszyło implementację interfejsu użytkownika. W porównaniu z narzędziami takimi jak Tailwind CSS czy Chakra UI, oferuje bardziej kompletne i ustrukturyzowane rozwiązanie\cite{retamozomaterialui}. 

Firebase Hosting wybrano z uwagi na prostotę wdrożeń oraz kompatybilność z pozostałymi usługami Firebase. W odróżnieniu od innych tego typu platform, np. Vercel, lepiej wpisywał się w przyjęty stos technologiczny \cite{uibakery2025vercel}.

\section{Warstwa backendowa}

Backend aplikacji opiera się na Firebase Authentication, bazie danych Firestore oraz dodatkowym API utworzonym w frameworku Flask i hostowanym na platformie Heroku. Firebase Authentication zapewnia bezpieczne, gotowe mechanizmy logowania i autoryzacji, bez konieczności wdrażania własnych rozwiązań, które byłyby podatne na błędy i trudniejsze w utrzymaniu\cite{firebase2025docs}.

Baza danych Firestore\cite{krzakfirestore} była naturalnym wyborem ze względu na ścisłe powiązanie z pozostałymi usługami Google wykorzystywanymi w projekcie. Istotną zaletą był również brak kosztów w podstawowym zakresie użytkowania. Alternatywne rozwiązania, takie jak PostgreSQL czy MongoDB, wymagałyby większego nakładu pracy administracyjnej, nie oferując w tym kontekście przewagi funkcjonalnej. 

Flask\cite{foxflask}, mikroframework napisany w języku Python, został wykorzystany do stworzenia prostego API pobierającego dane finansowe, głównie ze względu na jego lekkość, prostotę oraz bogaty ekosystem języka Python. Platforma Heroku\cite{mentormateheroku} zapewniła szybkie i bezproblemowe wdrożenie oraz stabilne działanie dzięki możliwości korzystania z GitHub Student Developer Pack\cite{github2025studentpack}. Inne darmowe rozwiązania hostingowe były mniej korzystne, ponieważ usypiały aplikację po okresie braku aktywności (ang. cold start\cite{payprocoldstart}), co skutkowałoby minutowym opóźnieniem przy pierwszym wywołaniu API, co było nieakceptowalne z perspektywy użytkownika.

\section{Pozostałe serwisy oraz narzędzia}

W projekcie wykorzystano system kontroli wersji Git oraz platformę GitHub, które zapewniły możliwość kontroli zmian oraz automatycznego wdrażania aplikacji za pomocą GitHub Actions. Jest to branżowy standard\cite{easyredminegithub},  który gwarantuje stabilny i transparentny proces rozwoju oprogramowania.

Biblioteka \texttt{yfinance} umożliwiła darmowe pobieranie danych finansowych dzięki wykorzystaniu nieoficjalnego API Yahoo Finance\cite{geeksforgeeksyfinance} . Wybór ten wynikał z konieczności użycia rozwiązania pozwalającego na pozyskiwanie aktualnych danych dotyczących aktywów notowanych na giełdach innych niż amerykańska, co w przypadku aplikacji skierowanej do polskiego inwestora było szczególnie istotne.

Serwis ExchangeRateAPI\cite{exchangerate} został wykorzystano do pobierania kursów walut ze względu na wysoki limit darmowych zapytań oraz łatwość w użytkowaniu. Usługa ta zapewnia stabilne i aktualne dane niezbędne do przeliczania wartości w portfelach wielowalutowych.  Alternatywnym, również bezpłatnym rozwiązaniem jest serwis Frankfurter. Ostatecznie wybrano jednak ExchangeRateAPI, ponieważ jest bardziej niezawodny.