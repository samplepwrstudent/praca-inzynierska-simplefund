\Chapter{Testowanie aplikacji}

Testy aplikacji zostały przeprowadzone w dwóch komplementarnych wymiarach. Pierwszy obejmował testy funkcjonalne, zrealizowane z wykorzystaniem frameworka Cypress, których celem była weryfikacja poprawności działania poszczególnych funkcjonalności systemu w typowych scenariuszach użytkowych.

Drugim wymiarem była analiza jakościowa i wydajnościowa, wykonana za pomocą narzędzia Google Lighthouse. Lighthouse umożliwia automatyczną ocenę kluczowych parametrów aplikacji webowej, obejmując m.in. efektywność renderowania, stabilność wizualną, szybkość interakcji, dostępność oraz zgodność z dobrymi praktykami projektowania interfejsów. Narzędzie dokonuje pomiarów zarówno aspektów technicznych, jak i czynników wpływających na percepcję i komfort użytkownika, w tym metryk powiązanych z Web Vitals.
\section{Testy end-to-end w Cypress}

Cypress jest nowoczesnym frameworkiem do automatyzacji testów aplikacji webowych, charakteryzującym się bezpośrednim wykonywaniem testów w przeglądarce, co zapewnia wysoką wierność odwzorowania rzeczywistych interakcji użytkownika. \cite{browserstack2025cypressreact}
Istotnym elementem architektury testów jest mechanizm zapewniający izolację poszczególnych przypadków testowych. 

W tym celu każdy test poprzedzony jest procedurą logowania użytkownika testowego, a po jego zakończeniu następuje procedura wylogowania. Takie podejście gwarantuje, że każdy test rozpoczyna się w znanym stanie aplikacji, eliminując potencjalne zależności między testami oraz zapewniając ich powtarzalność. 


\begin{listing}[h!]
\begin{minted}[fontsize=\small, linenos]{javascript}
it("add stock modal should show", () => {
    cy.get('[aria-label="dodaj operację"]').click();
    cy.get('[aria-label="Dodaj akcje"]').click();
    cy.contains("Dodaj Operację - Akcje");
    cy.contains("Anuluj").click();
})
\end{minted}
\caption{Test weryfikujący wyświetlanie modala dodawania operacji}
\label{lst:cypress-modal-test}
\end{listing}

Testy weryfikują poprawne renderowanie najważniejszych sekcji aplikacji: tabel operacji, dostępność modali służących do dodawania nowych operacji, prezentację statystyk portfela oraz renderowanie wykresów.



W teście przedstawionym w Listingu~\ref{lst:cypress-modal-test} symulowane jest kliknięcie przycisku dodawania operacji, następnie wybór typu aktywa (akcje), a na końcu weryfikacja obecności odpowiednich elementów interfejsu oraz zamknięcie modala. Tego rodzaju testy pozwalają na automatyczną weryfikację, że komponenty dialogowe poprawnie się otwierają, wyświetlają odpowiednie treści i mogą być zamykane przez użytkownika. 

Framework Cypress umożliwia dynamiczne monitorowanie przebiegu testów oraz analizę ich wyników w przejrzystym interfejsie graficznym, co zostało zobrazowane na Rysunku \ref{fig:cypress_action}.



\begin{figure}[h!]
    \centering
    \includegraphics[width=0.9\linewidth]{images/cypress_w_trakcie_testow.png}
    \caption{Widok Cypress w trakcie przeprowadzania testów}
    \label{fig:cypress_action}
\end{figure}

\section{Audyt wydajności z Google Lighthouse}

Google Lighthouse to zautomatyzowane narzędzie open-source służące do audytu jakości stron internetowych \cite{chrome2025lighthouse}. Przeprowadza ono kompleksową analizę aplikacji webowych w czterech kategoriach: wydajności (Performance), dostępności (Accessibility), najlepszych praktyk (Best Practices) oraz optymalizacji pod kątem wyszukiwarek (SEO - Search Engine Optima).

\begin{figure}[h!]
    \centering
    \includegraphics[width=0.5\linewidth]{images/lighthouse_overview.png}
    \caption{Wyniki audytu Google Lighthouse}
    \label{fig:lighthose_wyniki}
\end{figure}


Lighthouse generuje szczegółowe raporty zawierające zarówno numeryczne oceny (Rysunek \ref{fig:lighthose_wyniki}), jak i konkretne rekomendacje dotyczące poprawy jakości aplikacji. Audyt został przeprowadzony dla witryny \texttt{simplefund.me} (nie w środowisku lokalnym) z wcześniej zalogowanym użytkownikiem  na urządzeniu typu desktop o rozdzielczości 1280×720 pikseli. Jego wyniki zaprezentowano w tabeli \ref{tab:lighthouse-results}.

\begin{table}[h!]
\centering
\begin{tabular}{|l|c|c|}
\hline
\textbf{Kategoria} & \textbf{Wynik} & \textbf{Ocena} \\ \hline
Wydajność (Performance)        & 72/100  & Wymagający poprawy \\ \hline
Dostępność (Accessibility)     & 91/100  & Dobry \\ \hline
Najlepsze praktyki (Best Practices) & 100/100 & Dobry \\ \hline
Optymalizacja SEO              & 92/100  & Dobry \\ \hline
\end{tabular}
\caption{Wyniki audytu Google Lighthouse}
\label{tab:lighthouse-results}
\end{table}


Wynik w kategorii wydajności, choć mieszczący się w przedziale zadowalającym, wskazuje na obszary wymagające optymalizacji, szczególnie w zakresie czasu ładowania kluczowych elementów strony oraz stabilności wizualnej interfejsu. Jednocześnie pozostałe kategorie: dostępność, najlepsze praktyki oraz SEO uzyskały bardzo wysokie wyniki co wskazuje na zgodność z wytycznymi Google oraz ogólną poprawność techniczną aplikacji w tych obszarach.


\subsection*{Core Web Vitals}

Core Web Vitals (Rysunek \ref{fig:web_vitals_zakres}) to zestaw kluczowych metryk opracowanych przez Google, które umożliwiają ocenę jakości doświadczenia użytkownika w kontekście wydajności stron internetowych i aplikacji webowych  \cite{webdev2020vitals}. Metryki te koncentrują się na trzech fundamentalnych aspektach interakcji użytkownika ze stroną: czasie ładowania istotnej zawartości, stabilności wizualnej układu oraz responsywności interfejsu.

\begin{figure}[h]
    \centering
    \includegraphics[width=1\linewidth]{images/web_core_vitals.png}
    \caption{Core Web Vitals}
    \label{fig:web_vitals_zakres}
\end{figure}


Largest Contentful Paint (LCP) mierzy czas potrzebny na wyrenderowanie największego elementu treści widocznego dla użytkownika, co bezpośrednio wpływa na postrzeganą szybkość działania strony. 

Cumulative Layout Shift (CLS) określa poziom nieoczekiwanych przesunięć elementów wizualnych podczas ładowania strony, a tym samym stabilność układu i komfort korzystania z serwisu. 

Interaction to Next Paint (INP), metryka rozwinięta w kontekście najnowszych standardów Core Web Vitals, ocenia pełną responsywność interfejsu poprzez mierzenie opóźnienia między interakcją użytkownika a momentem, w którym interfejs reaguje wizualnie, co ma kluczowe znaczenie dla płynności i intuicyjności obsługi strony. \cite{optimizing_core_web_vitals_2025}



Zestawienie tych wskaźników pozwala nie tylko na szczegółową analizę wydajności strony z perspektywy użytkownika, lecz także stanowi istotny element algorytmów rankingowych wyszukiwarek, co sprawia, że optymalizacja Core Web Vitals ma bezpośrednie przełożenie zarówno na komfort użytkowania, jak i na widoczność serwisu w wynikach wyszukiwania.
 
Wyniki wskaźników LCP oraz CLS zostały bezpośrednio pobrane z raportu Lighthouse. Metrykę INP zbadano lokalnie, ponieważ nie jest ona dostępna w raporcie. Tabela~\ref{tab:webvitals} przedstawia uzyskane wartości poszczególnych wskaźników wraz z ich ocenami. Poszczególne oceny opierają się na oficjalnych progach jakości definiowanych przez Google.

\begin{table}[h!]
\centering
\begin{tabular}{|l|c|c|}
\hline
\textbf{Metryka} & \textbf{Wynik} & \textbf{Ocena} \\ \hline
Largest Contentful Paint (LCP) & 1488 ms & Wymagający poprawy \\ \hline
Cumulative Layout Shift (CLS)   & 0.339 & Słaby \\ \hline
Interaction to Next Paint (INP) & 128 ms & Dobry \\ \hline
\end{tabular}
\caption{Wyniki Core Web Vitals aplikacji}
\label{tab:webvitals}
\end{table}

Najistotniejszym problemem zidentyfikowanym podczas audytu okazała się wartość Cumulative Layout Shift wynosząca 0.34, która znacząco przekracza zalecany próg 0.1. Wysoka wartość CLS oznacza, że w trakcie ładowania strony występują nieprzewidywane przesunięcia elementów interfejsu, co negatywnie wpływa na doświadczenie użytkownika, mogąc prowadzić do przypadkowych kliknięć w niewłaściwe elementy. Analiza przyczyn tego zjawiska wskazała na wykres kołowy portfela jako głównego sprawcę problemu - komponent ten, ładowany asynchronicznie wraz z danymi z Firestore, powoduje przesunięcie pozostałych elementów układu po swoim wyrenderowaniu.

Lighthouse zidentyfikował także możliwości optymalizacji obrazu logo aplikacji, sugerując zastosowanie atrybutu \texttt{fetchpriority="high"} oraz zapewnienie jego wykrywalności bezpośrednio w dokumencie HTML, co pozwoliłoby na wcześniejsze rozpoczęcie pobierania tego krytycznego zasobu.