\chapter{Wnioski i kierunki dalszego rozwoju}

W ramach realizacji projektu udało się zrealizować większość założeń funkcjonalnych przewidzianych na etapie projektowym. Aplikacja umożliwia użytkownikowi zarządzanie portfelem inwestycyjnym poprzez dodawanie różnych typów aktywów, takich jak akcje, lokaty bankowe oraz depozyty gotówkowe. Zrealizowano pełny system uwierzytelniania, obejmujący zarówno tradycyjne logowanie, jak i logowanie za pośrednictwem konta Google, co zwiększa wygodę użytkowania oraz bezpieczeństwo dostępu. System poprawnie przetwarza i importuje dane z platformy XTB, umożliwiając integrację z rzeczywistym środowiskiem inwestycyjnym. 

Użytkownik uzyskuje dostęp do zestawu przejrzystych wizualizacji przedstawiających strukturę portfela oraz szczegółowe informacje na temat poszczególnych inwestycji. Aplikacja wylicza również istotne wskaźniki finansowe, takie jak ROI, CAGR czy bilans całkowity, które wspierają podejmowanie decyzji inwestycyjnych. Wdrożono ponadto automatyczne pobieranie aktualnych notowań instrumentów finansowych oraz ich przeliczenie na polskie złote, co umożliwia bieżące monitorowanie wartości portfela, niezależnie od waluty akcji. Interfejs użytkownika wyposażono w możliwość przełączania pomiędzy motywem jasnym a ciemnym, a wszystkie formularze poddawane są walidacji, co minimalizuje ryzyko wprowadzania niepoprawnych danych. Dane użytkowników są w sposób bezpieczny, z wykorzystaniem mechanizmów Firestore Security Rules.

Nie wszystkie zakładane cele udało się jednak spełnić w pełnym zakresie. W szczególności problematyczne okazało się osiągnięcie docelowej wartości wskaźnika Cumulative Layout Shift (CLS), który w finalnej wersji aplikacji przekracza rekomendowany poziom 0,25. Wynik pozostaje powyżej założonych progów jakościowych, co wskazuje na potrzebę dalszej poprawy warstwy prezentacyjnej.

\section*{Realny scenariusz użycia}
Przedstawiona aplikacja posiada szereg możliwości praktycznego wykorzystania. Rozważmy scenariusz, w którym użytkownik posiada w swoim portfelu u brokera XTB akcje polskich spółek PZU oraz CD Projekt Red. Użytkownik loguje się do systemu za pomocą konta Google. Dodatkowo dysponuje on oszczędnościami w wysokości 1000 PLN oraz lokatą bankową w iPKO, oprocentowaną na poziomie 1,5\%. Jego celem jest monitorowanie wartości całego portfela inwestycyjnego, kontrolowanie proporcji pomiędzy klasami aktywów, w szczególności utrzymywanie lokat na poziomie około 45\% portfela, a także dążenie do osiągnięcia rocznego zwrotu przekraczającego 4\%.

\begin{figure}[h]
    \centering
    \includegraphics[width=0.8\linewidth]{images/porftel_realny_scenariusz.png}
    \caption{Portfel użytkownika z realnego scenariusza użycia}
    \label{fig:portfel_scenariusz}
\end{figure}

Aplikacja umożliwia realizację tych założeń (Rysunek \ref{fig:portfel_scenariusz}) poprzez bieżące śledzenie zmian wartości poszczególnych aktywów, automatyczne aktualizowanie notowań oraz prezentowanie przejrzystych wykresów przedstawiających strukturę portfela. Dzięki obliczanym wskaźnikom inwestycyjnym użytkownik może łatwo ocenić, czy jego inwestycje zbliżają się do zakładanych celów finansowych, a w razie potrzeby odpowiednio dostosować strategię. Tym samym system stanowi praktyczne narzędzie wspierające proces świadomego i uporządkowanego zarządzania portfelem inwestycyjnym.

\section*{Możliwości ulepszeń i dalszego rozwoju}

Ocena zrealizowanego rozwiązania wskazuje obszary, w których aplikacja mogłaby zostać udoskonalona. Po pierwsze, warto skupić się na optymalizacji interfejsu użytkownika w celu zapewnienia, aby wskaźnik CLS mieścił się w zalecanych przedziałach.

Kolejnym istotnym rozszerzeniem funkcjonalnym byłoby wprowadzenie obsługi obligacji skarbowych z aktualną ofertą Narodowego Banku Polskiego. Ten typ instrumentu finansowego jest szczególnie istotny dla początkujących inwestorów w Polsce, a jego brak w obecnej wersji aplikacji ogranicza kompletność prezentowanego portfela inwestycyjnego.

Ponadto warto rozważyć rozwinięcie modułu wizualizacji danych poprzez dodanie wykresów, umożliwiających analizę zmian portfela w czasie. Idealnym uzupełnieniem byłoby porównanie wyników inwestycyjnych z takimi wskaźnikami jak inflacja, wkład własny użytkownika czy benchmarki rynkowe, np. indeks WIG20. Tego rodzaju funkcjonalność pozwoliłaby użytkownikowi na lepszą ocenę efektywności inwestycji.
\newpage

Wdrożenie powyższych usprawnień nie tylko zwiększyłoby wartość praktyczną aplikacji, ale również przyczyniłoby się do poprawy jej użyteczności oraz zgodności z najlepszymi praktykami projektowania aplikacji webowych.


\section*{Podsumowanie}
W ramach realizacji projektu wykorzystanie języka TypeScript znacząco ułatwiło utrzymanie oraz dalszy rozwój kodu dzięki statycznemu typowaniu, który pozwala na wcześniejsze wykrywanie błędów oraz zwiększa czytelność i przewidywalność aplikacji. Równie istotne okazało się dzielenie komponentów React na mniejsze, wyspecjalizowane moduły, co pozytywnie wpłynęło na możliwość ich ponownego wykorzystania oraz modyfikacji. Biblioteka MUI zapewnia spójność wizualną interfejsu użytkownika, w szczególności poprzez wbudowane mechanizmy stylizacji, możliwość definiowania motywów oraz wygodną obsługę trybu jasnego i ciemnego. Jednocześnie należy zauważyć, że modyfikacja bardziej szczegółowych aspektów niektórych komponentów może być utrudniona, co w niektórych przypadkach może wymagać tworzenia własnych komponentów lub zastosowania bardziej elastycznej biblioteki.

Istotnym wnioskiem wynikającym z procesu projektowego jest konieczność dokładnej analizy źródeł danych na etapie planowania systemu. W omawianym projekcie pierwotne założenia dotyczące wykorzystania istniejących usług zewnętrznych (API finansowych) okazały się niewystarczające, co wymusiło dołączenie własnego backendu do już rozwijanej aplikacji frontendowej. W szczególności dostęp do bezpłatnych danych giełdowych jest znacząco ograniczony — większość usług oferuje darmowe informacje wyłącznie dla rynków amerykańskich. Z kolei API udostępniające dane dotyczące kursów walut są szeroko dostępne i zazwyczaj bezpłatne, co najprawdopodobniej wynika z niższej złożoności pozyskiwania oraz przetwarzania tego typu informacji w porównaniu z danymi giełdowymi.

Wykorzystanie usług Firebase i Firestore okazało się korzystnym rozwiązaniem ze względu na ich stabilność, bezpieczeństwo oraz prostotę konfiguracji. Zastosowanie Pythona wraz z frameworkiem Flask umożliwiło stworzenie lekkiego i efektywnego serwera realizującego jedną, ściśle określoną funkcję. Jednocześnie należy zauważyć, że darmowe usługi hostingowe dla backendu cechują się znacznymi ograniczeniami, takimi jak automatyczne usypianie instancji przy braku ruchu, co prowadzi do zauważalnych opóźnień przy pierwszym wywołaniu. W tym kontekście oferta Heroku dostępna w ramach programu dla studentów okazała się wystarczająca i dobrze dopasowana do potrzeb aplikacji.

Dodatkowym wyzwaniem zidentyfikowanym podczas procesu tworzenia była niespójność symboli instrumentów finansowych pomiędzy różnymi dostawcami danych. Przykładowo, ta sama akcja może występować pod odmiennymi oznaczeniami symboli na platformie XTB i w serwisie Yahoo Finance. Choć identyfikator ISIN \cite{isinofficial2025isin} mógłby stanowić jednoznaczny punkt odniesienia, nie jest on udostępniany w raportach XTB, co w praktyce uniemożliwia jego wykorzystanie w procesie integracji danych.

Ostatecznie realizacja projektu pozwoliła zidentyfikować zalety zastosowanych technologii, jak i ograniczenia wynikające z dostępności danych rynkowych oraz specyfiki usług zewnętrznych. Zdobyte doświadczenia mogą stanowić wartościową wskazówkę przy planowaniu i projektowaniu podobnych systemów w przyszłości.