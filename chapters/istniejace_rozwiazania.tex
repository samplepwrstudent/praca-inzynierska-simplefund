\Chapter{Istniejące rozwiązania}
Aby umieścić opracowany system w szerszym kontekście technologicznym, konieczne jest zbadanie istniejących rozwiązań, które ułatwiają monitorowanie i zarządzanie inwestycjami osobistymi. Obecny rynek nie oferuje wielu polskich narzędzi, które spełniałyby opisane funkcje. Skrupulatny przegląd takich narzędzi pozwala zidentyfikować zarówno ich mocne strony, jak i ograniczenia oraz nakreślić cele, które proponowany system ma osiągnąc, by wyróżniać się spośród już istniejących.

W poniższej sekcji przedstawiono przegląd i analizę wybranych narzędzi dostępnych dla inwestorów indywidualnych na rynku polskim, podkreślając ich podstawowe cechy, typowe przypadki użycia i potencjalne niedociągnięcia w kontekście współczesnych wymagań użytkowników.

\section{Serwis MyFund.pl}
Serwis myfund.pl stanowi polskojęzyczne narzędzie wspierające zarządzanie portfelem inwestycyjnym, oferujące rozbudowane możliwości analityczne zarówno dla pojedynczych spółek, jak i całych portfeli. Platforma umożliwia monitorowanie różnorodnych klas aktywów, obejmujących akcje krajowe i zagraniczne, fundusze inwestycyjne, obligacje, instrumenty pochodne, produkty strukturyzowane, waluty, a także nieruchomości i pożyczki społecznościowe. System automatycznie aktualizuje ceny instrumentów finansowych, zapewniając bieżące informacje niezbędne do podejmowania decyzji inwestycyjnych. Narzędzia analityczne obejmują zarówno analizę fundamentalną i techniczną spółek, jak również wielowymiarową ocenę portfela z uwzględnieniem zysku, ryzyka, ekspozycji walutowej oraz struktury aktywów w czasie. Platforma umożliwia tworzenie wielu portfeli i subportfeli, śledzenie strategii inwestycyjnych, a także subskrybowanie portfeli innych użytkowników.

W kontekście kosztów, myfund.pl funkcjonuje w modelu abonamentowym, oferując różne plany w przedziale od kilku do kilkunastu złotych miesięcznie, przy czym dostęp do pełnej funkcjonalności wymaga opłacania subskrypcji. Pomimo szerokiego zakresu funkcji i narzędzi, interfejs serwisu pozostaje przestarzały i mało intuicyjny, co w połączeniu z mnogością dostępnych opcji może prowadzić do trudności w obsłudze, zwłaszcza w przypadku początkujących inwestorów. Bogactwo funkcjonalności, choć niewątpliwie zaletą dla osób doświadczonych, może generować poczucie przytłoczenia i zwiększać ryzyko nieefektywnego wykorzystania dostępnych narzędzi.

Podsumowując, myfund.pl stanowi wszechstronne i zaawansowane narzędzie wspierające zarządzanie inwestycjami, którego potencjał najlepiej wykorzystają użytkownicy średniozaawansowani i zaawansowani, wymagający kompleksowej analizy portfela i instrumentów finansowych. Jednocześnie jego charakterystyka oraz sposób prezentacji danych wskazują, że dla początkujących inwestorów konieczne jest poświęcenie dodatkowego czasu na naukę obsługi oraz świadome planowanie korzystania z funkcji serwisu.
\begin{figure}[H]
    \centering
    \includegraphics[width=0.8\linewidth]{images/myfund3.png}
    \caption{Kokpit serwisu myfund.pl}
    \label{fig:placeholder}
\end{figure}

\section{Arkusz do monitorowania inwestycji serwisu Inwestomat.eu}
Arkusz Inwestomat.eu to w pełni darmowe narzędzie przeznaczone do monitorowania portfela inwestycyjnego, udostępniane jako plik Google Sheets, który użytkownik może skopiować i samodzielnie obsługiwać. Jego kluczową zaletą jest całkowity brak kosztów korzystania, co znacząco obniża barierę wejścia. Jednocześnie arkusz oferuje rozbudowany zakres funkcjonalności, obejmujący m.in. możliwość podziału portfela na konta i klasy aktywów, automatyczne odświeżanie kursów za pomocą makr Google Apps Script, zapisywanie historii wartości portfela oraz obsługę różnych typów instrumentów, w tym akcji, funduszy ETF, obligacji skarbowych i walut. Wbudowany mechanizm FIFO umożliwia prawidłowe obliczanie zysków podatkowych, a dodatkowe funkcje związane z uwzględnianiem inflacji i narosłych odsetek zwiększają użyteczność arkusza w kontekście obsługi polskich obligacji detalicznych.

Pomimo licznych zalet korzystanie z arkusza wiąże się z istotnymi ograniczeniami. Przede wszystkim narzędzie to nie jest intuicyjne i wymaga poświęcenia czasu na zapoznanie się z jego strukturą oraz sposobem działania, co może zniechęcać początkujących inwestorów niechętnych do inwestowania wysiłku w obsługę rozwiązania, którego długoterminowej użyteczności nie są pewni. Konfiguracja arkusza wymaga wykonania szeregu kroków, przy których stosunkowo łatwo o pomyłki lub nieporozumienia, zwłaszcza wśród mniej zaawansowanych użytkowników. Dodatkowo interfejs oparty na arkuszu kalkulacyjnym cechuje się surową, mało atrakcyjną estetyką, co może z kolei zniechęcać osoby przyzwyczajone do nowoczesnych, graficznych aplikacji inwestycyjnych.

Podsumowując, arkusz Inwestomat.eu stanowi wartościowe i funkcjonalne narzędzie przede wszystkim dla użytkowników, którzy nie chcą ponosić kosztów zakupu oprogramowania oraz są wystarczająco wytrwali i obeznani z obsługą Google Sheets, aby samodzielnie przejść przez proces konfiguracji i korzystać z dostępnej dokumentacji.
\begin{figure}[h]
    \centering
    \includegraphics[width=1\linewidth]{images/inwestomat_excel.png}
    \caption{Fragmetn arkusza serwisu Inwestomat.eu}
    \label{fig:placeholder}
\end{figure}

\section{Luka rynkowa oraz potrzeba nowego systemu}
Analiza istniejących rozwiązań wskazuje, że na polskim rynku brakuje narzędzia do monitorowania inwestycji, które byłoby jednocześnie intuicyjne, przyjazne dla początkujących użytkowników oraz całkowicie darmowe. Najbardziej rozbudowane dostępne systemy, takie jak myfund.pl, oferują ogromne możliwości analityczne i szeroki zakres funkcji, jednak ich interfejs cechuje się wysokim poziomem złożoności oraz mało nowoczesną estetyką. Dla początkujących inwestorów może to stanowić barierę wejścia, prowadząc do poczucia przytłoczenia i trudności w efektywnym korzystaniu z platformy.

Z kolei rozwiązania o charakterze darmowym, takie jak arkusz Inwestomat.eu, choć funkcjonalne i elastyczne, wymagają od użytkownika samodzielnej konfiguracji oraz dobrej znajomości arkuszy kalkulacyjnych. Surowy wygląd interfejsu oraz konieczność ręcznego wykonywania wielu kroków konfiguracyjnych mogą zniechęcać osoby poszukujące prostego i wygodnego narzędzia, które działa bez wcześniejszej złożonej konfiguracji.

W konsekwencji na rynku istnieje wyraźna luka — brakuje rozwiązania, które łączyłoby prostotę obsługi, nowoczesny i czytelny interfejs, automatyzację podstawowych procesów oraz brak kosztów korzystania. Proponowany system ma na celu wypełnienie tej przestrzeni, oferując intuicyjną, estetyczną i całkowicie darmową platformę, przeznaczoną głównie dla początkujących lub pasywnych inwestorów, którzy oczekują narzędzia łatwego w codziennym użytkowaniu przy zachowaniu kluczowych funkcjonalności niezbędnych do monitorowania portfela.
