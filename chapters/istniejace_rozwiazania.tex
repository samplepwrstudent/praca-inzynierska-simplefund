\Chapter{Istniejące rozwiązania}
Aby umieścić opracowany system w szerszym kontekście technologicznym, konieczne jest zbadanie istniejących rozwiązań, które ułatwiają monitorowanie i zarządzanie inwestycjami osobistymi. Obecny rynek nie oferuje wielu polskich narzędzi, które spełniają opisane funkcje. Przegląd takich narzędzi pozwala zidentyfikować zarówno ich mocne strony, jak i ograniczenia oraz nakreślić cechy, które proponowany system miałby posiadać, aby wyróżniał się spośród już istniejących.

W poniższej sekcji przedstawiono przegląd i analizę wybranych narzędzi dostępnych dla inwestorów indywidualnych na rynku polskim, podkreślając ich podstawowe cechy, typowe przypadki użycia i potencjalne niedoskonałości w kontekście współczesnych wymagań użytkowników.

\section*{Serwis myfund.pl}
Serwis \textit{myfund.pl}\cite{myfund2025widget} stanowi polskojęzyczne narzędzie wspierające zarządzanie portfelem inwestycyjnym, oferujące rozbudowane możliwości analityczne zarówno dla pojedynczych spółek, jak i całych portfeli. Platforma umożliwia monitorowanie różnorodnych klas aktywów, obejmujących akcje krajowe i zagraniczne, fundusze inwestycyjne, obligacje, instrumenty pochodne, waluty, a także nieruchomości. System automatycznie aktualizuje ceny instrumentów finansowych, zapewniając bieżące informacje niezbędne do podejmowania decyzji inwestycyjnych. Narzędzia analityczne obejmują zarówno analizę fundamentalną, jak i techniczną spółek, a także wielowymiarową ocenę portfela, z uwzględnieniem zysku, ryzyka, ekspozycji walutowej oraz struktury aktywów w czasie. Platforma umożliwia tworzenie wielu portfeli i subportfeli, śledzenie strategii inwestycyjnych, a także subskrybowanie portfeli innych użytkowników.

W kontekście kosztów, myfund.pl funkcjonuje w modelu abonamentowym, oferując różne plany w przedziale od kilku do kilkunastu złotych miesięcznie, przy czym dostęp do pełnej funkcjonalności wymaga opłacania subskrypcji. Pomimo szerokiego zakresu funkcji i narzędzi, interfejs serwisu pozostaje przestarzały i mało intuicyjny, co w połączeniu z mnogością dostępnych opcji może prowadzić do trudności w obsłudze, zwłaszcza w przypadku początkujących inwestorów. Bogactwo funkcjonalności, choć niewątpliwie jest zaletą dla osób doświadczonych, może przytłaczać poziomem skomplikowania początkujących użytkowników.

Podsumowując, myfund.pl stanowi wszechstronne i zaawansowane narzędzie wspierające zarządzanie inwestycjami, którego potencjał najlepiej wykorzystają zaawansowani użytkownicy, wymagający kompleksowej analizy portfela oraz instrumentów finansowych. Jednocześnie jego charakterystyka oraz sposób prezentacji danych wskazują, że dla początkujących inwestorów konieczne jest poświęcenie dodatkowego czasu na naukę obsługi serwisu.

\begin{figure}[H]
    \centering
    \includegraphics[width=0.8\linewidth, height=10cm, keepaspectratio]{images/myfund3.png}
    \caption{Kokpit serwisu myfund.pl}
    \label{fig:myfund}
\end{figure}

\section*{Arkusz do monitorowania inwestycji serwisu Inwestomat.eu}
Arkusz dostarczony przez serwis Inwestomat.eu\cite{inwestomat2025arkusz} to w pełni darmowe narzędzie przeznaczone do monitorowania portfela inwestycyjnego, udostępniane jako plik Google Sheets, który użytkownik może skopiować i samodzielnie obsługiwać. Jego zaletą jest całkowity brak kosztów korzystania, co znacząco obniża barierę wejścia. Jednocześnie arkusz oferuje rozbudowany zakres funkcjonalności, obejmujący m.in. możliwość podziału portfela na konta oraz klasy aktywów, automatyczne odświeżanie kursów, zapisywanie historii wartości portfela oraz obsługę różnych typów instrumentów, w tym akcji, funduszy ETF, obligacji skarbowych i walut.

Pomimo licznych zalet, korzystanie z arkusza wiąże się z istotnymi ograniczeniami. Przede wszystkim narzędzie to nie jest intuicyjne i wymaga poświęcenia czasu na zapoznanie się z jego strukturą oraz sposobem działania, co może zniechęcać początkujących inwestorów do nauki obsługi tego rozwiązania. Konfiguracja arkusza wymaga wykonania szeregu kroków, przy których stosunkowo łatwo o pomyłki lub nieporozumienia. Dodatkowo interfejs oparty na arkuszu kalkulacyjnym cechuje się surową, mało atrakcyjną estetyką, co może z kolei zniechęcać osoby przyzwyczajone do nowoczesnych, graficznych aplikacji inwestycyjnych.

Podsumowując, arkusz dostarczony przez serwis Inwestomat.eu stanowi wartościowe i funkcjonalne narzędzie, przede wszystkim dla użytkowników, którzy nie chcą ponosić kosztów zakupu oprogramowania oraz są wystarczająco obeznani z obsługą Google Sheets, aby samodzielnie przejść przez proces konfiguracji i korzystać z dostępnej dokumentacji.

\begin{figure}[h]
    \centering
    \includegraphics[width=0.7\linewidth]{images/inwestomat_excel.png}
    \caption{Fragment arkusza serwisu Inwestomat.eu}
    \label{fig:inwestomat}
\end{figure}

\section*{Potrzeba wdrożenia nowego rozwiązania}
Analiza istniejących rozwiązań wskazuje, że na polskim rynku brakuje narzędzia do monitorowania inwestycji, które byłoby jednocześnie intuicyjne, przyjazne dla początkujących użytkowników oraz całkowicie darmowe. Najbardziej rozbudowane dostępne systemy, takie jak myfund.pl, oferują ogromne możliwości analityczne i szeroki zakres funkcji, jednak ich interfejs cechuje się wysokim poziomem złożoności oraz mało nowoczesną estetyką. Z kolei rozwiązania o charakterze darmowym, takie jak arkusz serwisu Inwestomat.eu, choć funkcjonalne i elastyczne, wymagają od użytkownika samodzielnej konfiguracji oraz dobrej znajomości arkuszy kalkulacyjnych. Surowy wygląd interfejsu oraz konieczność ręcznego wykonywania wielu kroków konfiguracyjnych mogą zniechęcać osoby poszukujące prostego i wygodnego narzędzia, które działa bez złożonej konfiguracji.

W konsekwencji na rynku istnieje pewien niedobór, brakuje rozwiązania, które łączyłoby prostotę obsługi, nowoczesny i czytelny interfejs, automatyzację podstawowych procesów oraz brak kosztów korzystania. Proponowany system ma na celu wypełnienie tej przestrzeni, oferując intuicyjną, estetyczną i całkowicie darmową platformę, przeznaczoną głównie dla początkujących lub pasywnych inwestorów, którzy oczekują narzędzia łatwego w codziennym użytkowaniu, przy zachowaniu kluczowych funkcjonalności niezbędnych do monitorowania portfela.
